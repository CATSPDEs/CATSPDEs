\usepackage[utf8]{inputenc}
\usepackage[english,russian]{babel}
\usepackage[left=2cm, right=1cm, top=1cm, bottom=1cm]{geometry}
\usepackage[dvipsnames]{xcolor} % fancy colors
\usepackage{
	amsmath, amssymb, % math
	mathtools, xparse, % for norm, normL
	dutchcal, % lowercase mathcal font
	pdflscape, % portrait landscape (for large tables)
	bm, % bold for everything
	physics % partial derivatives \pdv{Q}{t}
}

%
% PICTURES
%
\usepackage{graphicx}
% sub figures / grids of pictures
% http://ctan.altspu.ru/macros/latex/contrib/caption/subcaption.pdf
\usepackage{subcaption} 
% http://tex.stackexchange.com/questions/49979/put-text-to-the-right-of-figures
\usepackage[rightcaption]{sidecap}
\sidecaptionvpos{figure}{c}
%\usepackage{showframe} % only for demo
\graphicspath{{img/}} % includegraphics path
\newcommand{\svginput}[1]{\input{img/#1}} % pdf_tex path
\newcommand{\svginputw}[2][\linewidth]{\def\svgwidth{#1}\input{img/#2}} % pdf_tex path
\newcommand{\includegraphicsw}[2][\linewidth]{\includegraphics[width=#1]{#2}}

%
% THEOREMS
%
\usepackage{amsthm}
\newtheorem{remark}{Замечание}
\theoremstyle{definition}
\newtheorem{definition}{Определение}

%
% BOXED EQNS
%
\usepackage{empheq}
% http://mirror.hmc.edu/ctan/macros/latex/contrib/mathtools/empheq.pdf
\newcommand*\widefbox[1]{\fbox{\hspace{1em}#1\hspace{1em}}}

%
% FONTS AND FORMATTING
%
% line interval
\renewcommand{\baselinestretch}{1.5} 
% font sizes for sectioning
% http://tex.stackexchange.com/questions/134031/how-to-adjust-the-size-and-placement-of-chapter-heading-in-report-class
\usepackage{titlesec}
\titleformat{\chapter}[display]{\centering \Large}{\chaptertitlename \ \thechapter}{0pt}{\bfseries}
\titlespacing*{\chapter}{0pt}{0pt}{15pt}
\titleformat*{\section}{\large\bfseries}
\titleformat*{\subsection}{\bfseries}
\titleformat*{\subsubsection}{\bfseries}

% in order to use subeqns within paragraphs: 
% http://tex.stackexchange.com/questions/27053/too-much-space-between-full-paragraph-and-subequations-env
%\usepackage{etoolbox}
%\preto\subequations{\ifhmode\unskip\fi}

%
% LINKS AND REFERENCES
%
\usepackage{hyperref}
\hypersetup{
	colorlinks,
	linkcolor={blue!60!black},
	citecolor={blue!60!black},
	urlcolor={blue!70!black}
}
%\newcommand{\catspdes}[1]{\href{https://github.com/CATSPDEs/CATSPDEs/tree/master/sln/#1}{\texttt{#1}}}
\newcommand{\catspdesp}[1]{\href{https://github.com/CATSPDEs/CATSPDEs/tree/master/sln/#1}{\texttt{#1}}}
\newcommand{\catspdest}[1]{\href{https://github.com/CATSPDEs/CATSPDEs/tree/master/sln/Tools/#1}{\texttt{#1}}}
\newcommand{\catspdestn}[2]{\href{https://github.com/CATSPDEs/CATSPDEs/tree/master/sln/Tools/#1}{\texttt{#2}}}
\newcommand{\catspdesm}[1]{\href{https://github.com/CATSPDEs/CATSPDEs/tree/master/sln/Tools/Mathematica}{\texttt{#1}}}

%
% TABLES
%
% table captions
\usepackage{caption} 
\captionsetup[table]{skip=15pt}
% relative width for cols of table
\usepackage{tabularx}
% smaller fonts for tabularx + paddings
\let\oldtabular\tabular
\renewcommand{\tabular}[1][1.5]{\def\arraystretch{#1}\footnotesize\oldtabular}
\let\oldtabularx\tabularx
\renewcommand{\tabularx}{\def\arraystretch{1.5}\footnotesize\oldtabularx}
% centering for individual cell
\newcommand{\Zccell}[1]{\centering\arraybackslash #1}
% centering for the whole col
\newcolumntype{C}{>{\Zccell}X}
\newcolumntype{L}{>{\raggedright\arraybackslash}X}
\newcolumntype{R}{>{\raggedleft \arraybackslash}X}
% multiple rows
\usepackage{multirow}

% highlight text w/ bg color
\newcommand{\Zhlight}[1]{\colorbox{Aquamarine}{#1}}
\newcommand{\Ztodo}[1]{\colorbox{Goldenrod}{#1}}

% symbols we often use
\newcommand{\catspdes}{\href{https://github.com/CATSPDEs/}{\texttt{CATS'\,PDEs}}}
\newcommand{\OmegaField}{{\Omega_{\text{field}}}}
\newcommand{\BMean}{{B_{\text{mean}}}}

\newcommand{\uIn}{\vect u_{\text{in}}}
\newcommand{\GammaIn}{\Gamma_{\text{in}}}
\newcommand{\GammaOut}{\Gamma_{\text{out}}}
\newcommand{\USpace}{\mathbb{\vect U}}
\newcommand{\PSpace}{\mathbb P}
\newcommand{\aForm}[2]{\mathbcal a(\vect #1, \vect #2)}
\newcommand{\bForm}[2]{\mathbcal b(\vect #1, #2)}
\newcommand{\lForm}[1]{\mathbcal l(\vect #1)}
\newcommand{\LSpace}[1][\Omega]{\mathbb L_2\left({#1}\right)}
\newcommand{\HSpace}[1][\Omega]{\mathbb H^1\left({#1}\right)}


% math commands for convinience
	% norm
	\NewDocumentCommand{\normL}{ s O{} m }{%
		\IfBooleanTF{#1}{\norm*{#3}}{\norm[#2]{#3}}_{L_2(\Omega)}%
	}
	% differentials
	\newcommand*\diff{\mathop{}\!\mathrm{d}}
	\newcommand*\Diff[1]{\mathop{}\!\mathrm{d^#1}}
	% span{…}, area, length
	\DeclareMathOperator{\spanop}{span}
	\newcommand{\spn}[1]{\spanop \left\{ #1 \right\}}
	\DeclareMathOperator{\area}{area \, of}
	\DeclareMathOperator{\length}{length \, of}
	\DeclareMathOperator{\sign}{sign}
	% bold vectors
	\newcommand{\vect}[1]{\boldsymbol{\mathbf{#1}}}